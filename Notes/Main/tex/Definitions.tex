\pagenumbering{gobble}
\chapter{Definitions}
\theoremstyle{definition}

\section{Functions of Several Variables}
\begin{mydef}
\normalfont A function \(f\) whose domain is \(\real^n\) or a subset of \(\real^n\), for \(n \ge 2\) and \(n \in \nat\), is called a function of several real variables.
\end{mydef}

\begin{mydef}
\normalfont For a function \(z = f(x, y)\): A vertical section is the graph of \(z = f(x, c)\) or \(z = f(c, y)\), for some constant \(c\). A level curve is the curve \(f(x, y) = c\), for some constant \(c\).
\end{mydef}

\section{Partial Differentiation}
\begin{mydef}
\normalfont The function \(f : \real^3 \to \real\) is said to have a partial derivative with respect to \(x\) at the point \((x_0, y_0, z_0)\) if the following limit exists
%
\[\lim_{\Delta x \to 0}{\frac{f(x_0 + \Delta x, y_0, z_0) - f(x_0, y_0, z_0)}{\Delta x}}\]
%
which is called the \textbf{partial derivative} of \(f\) with respect to \(x\) at the point \((x_0, y_0, z_0)\), denoted as
%
\[\frac{\partial f(x_0, y_0, z_0)}{\partial x} \equiv \lim_{\Delta x \to 0}{\frac{f(x_0 + \Delta x, y_0, z_0) - f(x_0, y_0, z_0)}{\Delta x}}\]
\end{mydef}

\paragraph{Geometric interpretation of partial derivative} The partial derivative 
\(f_x(a,b)\) is the slope of the tangent line to the curve \(f(x, b)\) at \(x = a\).

\begin{mythm}\normalfont
When a function has the second order continuous partial derivatives, the partial derivations of this function do not depend on the order with respect to the variables.
\end{mythm}

\section{L'Hospital's rule}
\begin{mythm}
\normalfont Let \(\lim{}\) stand for the limit of \(\lim_{x\to c}, \lim_{x\to +\infty}, \lim_{x\to -\infty}\)

If \(\lim{\frac{f'(x)}{g'(x)}}\) has a finite value or if the limit is \(\pm\infty\) then \(\lim{\frac{f(x)}{g(x)}} = \lim{\frac{f'(x)}{g'(x)}}\)
\end{mythm}

\section{Order of variables}
\subsection{Big-O notation}
\begin{mythm}
\normalfont Let \(\lim{}\) stand for the limit of: \(\lim_{x\to c}, \lim_{x\to+\infty}, \lim{x\to-\infty}\)

\[\lim{\frac{f(x)}{g(x)}=k}, k\ne 0, \text{ if and only if } f(x)=O(g(x))\]
\end{mythm}

\subsection{Little-o notation}
\begin{mythm}
\normalfont Let \(\lim{}\) stand for the limit of: \(\lim_{x\to c}, \lim_{x\to+\infty}, \lim_{x\to-\infty}\)

\[\lim{\frac{f(x)}{g(x)} = 0} \text{ if and only if } f(x) = o(g(x))\]
\end{mythm}

\section{Differentiable function of a single variable}
\begin{mythm}
\normalfont The function \(y=f(x)\) is called differentiable at \(x_0\) if
\[y-y_0 = f_x(x_0)(x-x_0) + o(x-x_0)\]
or
\[\Delta y = f_x(x_0)\Delta x + o(\Delta x)\]

we have
\[y - y_0 \approx f_x(x_0)(x-x_0)\]
\end{mythm}

\section{Differentiable functions with two variables}
\begin{mythm}\normalfont
The function \(z = f(x,y)\) is called differentiable at \((x_0, y_0)\) if
\[\Delta z = f_x(x_0, y_0)\Delta x + f_y(x_0, y_0)\Delta y + o(\rho)\]
or
\[z-z_0=f_x(x_0,y_0)(x-x_0)+f_y(x_0,y_0)(y-y_0)+o(\rho)\]

\textit{It is the equation for the tangent plane of the surface \(z = f(x,y)\) at \((x_0,y_0)\)}
\end{mythm}

\section{Differentiable Function}
\begin{mydef}
\normalfont The function \(f(x, y, z)\) is called differentiable at \((x_0, y_0, z_0)\) if \(\Delta f = f(x, y, z) - f(x_0, y_0, z_0)\) can be expressed as
%
\[\Delta f = f_x(x_0,y_0,z_0)\Delta x + f_y(x_0,y_0,z_0)\Delta y + f_z(x_0,y_0,z_0)\Delta z + o(\rho)\]
%
where \(\Delta x = x - x_0\), \(\Delta y = y - y_0\), \(\Delta z = z - z_0\), and \(\rho = \sqrt{(\Delta x)^2 + (\Delta y)^2 + (\Delta z)^2}\).

As \(\rho\) is infinitely small, we have
\[df = f_x(x_0,y_0,z_0)dx + f_y(x_0,y_0,z_0) + f_z(x_0,y_0,z_0)dz\]
\end{mydef}

\begin{mydef}
\normalfont Let \(f\) be a function of the variables \(x_1, x_2, \hdots, x_n\), i.e. 
%
\[f = f(x_1, x_2, \hdots, x_n)\]
%
where each \(x_j\) is a function of (some of) the variables \(t_1, t_2, \hdots, t_m\), i.e. 
%
\[x_j = x_j(t_1, t_2, \hdots, t_m), j = 1, 2, \hdots, n\]
%
If \(f\) and \(x_j\) are sufficiently smooth, then
%
\[\frac{\partial f}{\partial t_i} = \frac{\partial f}{\partial x_1}\frac{\partial x_1}{\partial t_i} + \frac{\partial f}{\partial x_2}\frac{\partial x_2}{\partial t_i} + \hdots + \frac{\partial f}{\partial x_n}\frac{\partial x_n}{\partial t_i}, i = 1,2,\hdots, m\]
%
\end{mydef}

\pagenumbering{arabic}


























